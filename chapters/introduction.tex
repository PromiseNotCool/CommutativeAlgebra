\chapter*{Introduction}

\addcontentsline{toc}{chapter}{Introduction}

The study and application of commutative rings with identity.
    \begin{enumerate}
        \item Commutative algebra in calculus. We have $\llC(\bbR) = \{\text{continuous functions }\bbR \to \bbR\}$ and $\llD(\bbR) = \{\text{dif and only iferentiable functions }\bbR \to \bbR\}$ are both commutative rings with identity.
        \item Commutative algebra in graph theory. Let $G$ be a finite simple graph with vertex set $V = \{v_1,\cdots,v_d\}$. The \emph{edge ideal} of $G$ is $\operatorname{I}(G) = \langle v_iv_j \mid v_iv_j \text{ is an edge in }G \rangle \leq K[v_1,\cdots,v_d]$. 
            \[\text{algebraic properties of $\operatorname{I}(G)$} \myrightleftarrows{\rule{0.8cm}{0cm}} \text{combinatorial properties of $G$}.\]
        \item Commutative algebra in combinatorics.  A simplicial complex $\Delta$ on $V$. Stanley-Reisner ideal $\operatorname{J}(\Delta) \leq K[v_1,\cdots,v_d]$. 
            \[\text{algebraic properties of $\operatorname{J}(\Delta)$} \myrightleftarrows{\rule{0.8cm}{0cm}} \text{combinatorics properties of $\Delta$}.\]
            Let $\llP$ be a poset and $\Delta(\llP) = ``\text{order complex of $\llP$''} = \{\text{chains in $\llP$}\}$. Study $\llP$ via $\operatorname{J}(\Delta(\llP))$. 
        \item Commutative algebra in number theory. Number theory is the study of solutions of polynomial equations over $\bbZ$. Given an intermediate field $\bbQ \subseteq K \subseteq \bbC$, let $R = \{\alpha \in K \mid \ex \text{monic }f \in \bbZ[x] \text{ s.t. } f(\alpha) = 0\}$. Then $\bbZ \subseteq R \subseteq K$ and $R$ is a subring of $K$. (Chapter 5)
        \item Commutative algebra in algebraic geometry. Algebraic geometry is the study of solution sets for systems of polynomial equations over fields. Let $k$ be a field, $f_1,\cdots,f_m \in k[X_1,\cdots,X_d]$, 
            \[V := \operatorname{V}(f_1,\cdots,f_m) = \{\underline x \in k^d \mid f_i(\underline x) = 0,\fa i = 1,\cdots,m\},\] 
            where $\operatorname{V}$ is for ``variety'', and 
            \[\operatorname{I}(V) = \{f \in k[X_1,\cdots,X_d \mid f(\underline x) = 0,\fa \underline x \in V\} \leq k[X_1,\cdots,X_d].\] 
            \[\text{algebraic properties of $\operatorname{I}(V)$} \myrightleftarrows{\rule{0.8cm}{0cm}} \text{geometric properties of $V$}.\]
    \end{enumerate}

    Why modules? Because in number theory, $R = \{\alpha \in K \mid \ex \text{monic }f \in \bbZ[x] \text{ s.t. } f(\alpha) = 0\}$ is a subring of $K$. \par
    \textbf{Challenge-exercise:} prove this by definition. Let $\alpha,\beta \in R$. Then there exist $f,g \in \bbZ[X]$ monic such that $f(\alpha) = 0 = f(\beta)$. Prove/construct monic polynomials $s,d,p \in \bbZ[X]$ such that $s(\alpha + \beta) = 0$, $d(\alpha-\beta) = 0$ and $p(\alpha \beta) = 0$. \par
    Proof is a straightforward application of modules. \par
    Why topology? To study geometry, need continuity. Let $V = \operatorname{V}(f_1,\cdots,f_m)$, $W = \operatorname{V}(g_1,\cdots,g_n)$ and $\phi: V \to W$. What does it mean for $\phi$ to be continuous if $K = \bbF_3$? Need a notion of open sets in $V$ and $W$.

