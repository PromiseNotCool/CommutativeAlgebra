\chapter{Primary Decomposition}

Let $R$ be a nonzero commutative ring with identity.

\begin{discussion}
    UFD's have prime factorization. In fact, it is ``if and only if''. \par 
    Aternative versions for non-UFD's.
    \begin{enumerate}
        \item irreducible factorizations: 
            \[
                \begin{array}{llll}
                    \text{\underline{Pros}} &&& \text{\underline{Cons}} \\
                    \text{familiar} &&& \text{don't necessarily exist} 
                \end{array}.
            \]
        \item 
            primary decompositions:
            \[
                \begin{array}{llll}
                    \text{\underline{Pros}} &&& \text{\underline{Cons}} \\
                    \text{exist, e.g., if $R$ is noetherian,} &&& \text{replace factorizations of elements} \\
                    \text{there exists more generators } &&& \text{with intersections of nice ideal} \\
                    \text{than just for principal ideal}
                \end{array}.
            \]
    \end{enumerate}
\end{discussion}

\begin{theorem}
    Let $R$ be a noetherian integral domain and $a \in R \setminus \{R^{\times} \cup 0\}$.
    \begin{enumerate}
        \item $a$ has an irreducible factor in $R$.
        \item There exist irreducible $b_1,\cdots,b_n \in R$ such that $a = b_1 \cdots b_n$.
    \end{enumerate}
\end{theorem}

\begin{proof}
    \begin{enumerate}
        \item Let $\Sigma = \{\langle b \rangle \neq R : b \mid a\}$. Since $\langle a \rangle \in \Sigma$, $\Sigma \neq \emptyset$. Since $R$ is noetherian, $\Sigma$ has a maximal element, say $\langle b \rangle$. Claim. $\langle b \rangle$ is irreducible. Note $b \neq 0$ since $a \neq 0$ and $b \mid a$, and $b \not \in R^{\times}$ since $\langle b \rangle \neq R$. Suppose $b = cd$ for some $c,d \in R$. Since $c \mid b \mid a$, we have $c \mid a$. Also, since $c \not \in R^{\times}$, $\langle c \rangle \in \Sigma$. Since $\langle b \rangle \subseteq \langle c \rangle \subseteq R$, $\langle b \rangle$ is maximal in $\Sigma$, we have $\langle cd \rangle = \langle b \rangle = \langle c \rangle$ or $\langle c \rangle = R$. Since $R$ is an integral domain, $d \in R^{\times}$ or $c \in R^{\times}$.
        \item If $a$ is irreducible, then done. Else there exists $b_1 \in R$ irreducible such that $b_1 \mid a$ and $a = b_1a_1$ for some $a_1 \in R$. If $a_1$ is irreducible, then done. Else there exists irreducible $b_2 \in R$ such that $b_2 \mid a_1$ and $a_1 = b_2a_2$ for some $a_2 \in R$. If $a_2$ is irreducible, then done and we have $\langle a \rangle \subsetneq \langle a_1 \rangle \subsetneq \langle a_2 \rangle$. By the ascending chain condition, the process will terminate in finite number of steps. \qedhere
    \end{enumerate}
\end{proof}

\begin{example}
    \begin{enumerate}
        \item 
            Let $k$ be a field and $A = k[X^{\bbR_{\geq 0}}] := \{\sum_{i \in \bbR_{\geq 0}}^{\text{finite}}a_iX^{i} \mid a_i \in k\}$. Let $\ffm = \langle X^{\bbR_{>0}} \rangle \lneq A$. Then $\ffm \in \operatorname{m-Spec}(R)$ and $A/\ffm \cong k$. Let $R = A_\ffm$. Since $X$ has no irreducible factors in $R$, $X$ has no irreducible factorization. Let $\epsilon > 0$ and $f \in R \setminus \{0\}$. Then $R \setminus \{R^{\times} \cup 0\} \ni X^{\epsilon} \cdot f = X^{\frac{\epsilon}{2}} \cdot X^{\frac{\epsilon}{2}} \cdot f$. So $X^{\epsilon}f$ is not irreducible. Thus, $R$ has no irreducible elements.
        \item 
            In $\bbZ_6$, we have $3^{2} = 3$, $2^{2} = 4$, $2^{3} = 2$.
    \end{enumerate}
\end{example}

\begin{definition}
    If $R$ satisfies the conclusion of Theorem 4.2(b), then $R$ is \emph {atomic}.
\end{definition}

\begin{lemma}[Nakayama's Lemma]
    Let $I,J \leq R$ such that $I \subseteq \operatorname{Jac}(R)$ and $J$ is finitely generated. If $J = IJ$, then $J = 0$.
\end{lemma}

\begin{proof}
    Let $n$ be the minimum number of generators of $J$. Suppose $n \geq 2$. Since $J$ is finitely generated, $IJ = J = \langle x_1,\cdots,x_n \rangle$. Then $x_n \in IJ$ and so $x_n = \sum_{i=1}^{n}a_ix_i$ for some $a_1,\cdots,a_n \in I$. Hence $x_n(1-a_n) = \sum_{i=1}^{n}a_ix_i$. Since $a_n \in I \in \operatorname{Jac}(R)$, $1-a_n \in R^{\times}$ by Proposition 1.29. So $x_n \in \langle X_1,\cdots,X_{n-1} \rangle$, contradicting minimality of $n$. Hence $n = 1$ or $0$. If $n = 1$, similarly, we have $x_1(1-a_1) = 0$ for some $a_1 \in I$ with $1-a_1 \in R^{\times}$, so $x_1 = 0$, a contradiction. Thus, $n = 0$.
\end{proof}

\begin{lemma}
    Let $(R,\ffm)$ be local and $0 \neq b = cd$ with $b,c,d \in R$ such that $\langle b \rangle = \langle c \rangle$. Then $d \in R^{\times}$.
\end{lemma}

\begin{proof}
    Since $b = cd$ and $\langle b \rangle = \langle c \rangle$, we have $\langle c \rangle = \langle b \rangle = \langle cd \rangle = \langle d \rangle \langle c \rangle$. Suppose $d \not \in R^{\times}$. Then $\langle d \rangle \subseteq \ffm = \operatorname{Jac}(R)$. So by Lemma 4.5, $c = 0$. Hence $b = cd = 0$, a contradiction. Thus, $d \in R^{\times}$.
\end{proof}

\begin{theorem}
    Let $(R,\ffm)$ be local and noetherian. Let $a \in R \setminus \{R^{\times} \cup 0\}$. 
    \begin{enumerate}
        \item $a$ has an irreducible factoror in $R$.
        \item $a = b_1 \cdots b_n$ for some irreducible elements $b_1,\cdots,b_n \in R$.
    \end{enumerate}
\end{theorem}

\begin{proof}
    Similar to the proof of Theorem 4.2.
\end{proof}

\begin{discussion}
    Let $R$ be noetherian (local or a domain) and $a \in R \setminus \{R^{\times} \cup 0\}$ with irreducible factorization $a = b_1 \cdots b_n$. Then $\operatorname{V}(a) = \operatorname{V}(b_1 \cdots b_n) = \operatorname{V}(b_1) \cup \cdots \operatorname{V}(b_n)$, which are not irreducible components since $\langle b_1 \rangle, \cdots, \langle b_n \rangle$ are not minimal primes of $R$.
\end{discussion}

\begin{example}
    Let $R = \frac{k[X,Y,Z]_{(X,Y,Z)}}{(X^{2}-YZ)}$ or $R = \frac{k\llbracket X,Y,Z \rrbracket }{(X^{2}-YZ)}$. Then $R$ is a local, noetherian and integral domain. Let $x = \overbar{X} \in R$, which is irreducible. Since $(x,z) \in \operatorname{V}(x) \setminus \operatorname{V}(y)$, $\operatorname{V}(x) \neq \operatorname{V}(y)$. Since $(x,y) \in \operatorname{V}(x) \setminus \operatorname{V}(z)$, $\operatorname{V}(x) \neq \operatorname{V}(z)$. Also, since $\operatorname{V}(x) = \operatorname{V}(x^{2}) = \operatorname{V}(yz) = \operatorname{V}(y) \cup \operatorname{V}(z)$, we have $\operatorname{V}(x)$ is not irreducible in $\operatorname{Spec}(R)$. 

\end{example}

\noindent Primary decomposition does the job.

\begin{definition}
    An ideal $\ffq \lneq R$ is \emph{primary} if $xy \in \ffq$ with $x,y \in R$, then $x \in \ffq$ or $y \in \operatorname{rad}(\ffq)$, i.e., if $\overbar{x} \overbar{y} = 0$ with $\overbar{x}, \overbar{y} \in R/\ffq$, then $\overbar{x} = 0$ or $\overbar{y} \in \operatorname{Nil}(R/\ffq)$, i.e., if $xy \in \ffq$ with $x,y \in R$, then $x \in \ffq$ or $y \in \ffq$ or $x,y \in \operatorname{rad}(\ffq)$.
\end{definition}

\begin{example}
    \begin{enumerate}
        \item 
            If $\ffp \in \operatorname{Spec}(R)$, then $\ffp$ is primary since $\operatorname{rad}(\ffp) = \ffp$.
        \item 
            If $\ffm \in \operatorname{Spec}(R)$ and $\ffq \leq R$ such that $\ffm^{n} \subseteq \ffq \subseteq \ffm$ for some $n \geq 1$, then $\ffq$ is primary. In particular, $\ffm^{n}$ is primary for $n \geq 1$.
            \begin{proof}
                Let $xy \in \ffq \subseteq \ffm$ with $x,y \in R$. Assume $y \not\in \operatorname{rad}(\ffq)$. Since $\ffm \in \operatorname{Spec}(R)$, $x \in \ffm$ or $y \in \ffm$. Since $\operatorname{rad}(\ffm) = \operatorname{rad}(\ffm^{n}) \subseteq \operatorname{rad}(\ffq) \subseteq \operatorname{rad}(\ffm) = \ffm$, we have $\operatorname{rad}(\ffq) = \ffm$. Since $\ffm \in \operatorname{m-Spec}(R)$ and $y \not \in \operatorname{rad}(\ffq) = \ffm$, $\langle y,\ffm \rangle = R$. As in Proposition 1.46 (b), we can show $\langle y,\ffm^{n} \rangle = R$ by Proposition 1.39(a). So $1 = zy + \alpha$ for some $z \in R$ and $\alpha \in \ffm^{n} \subseteq \ffq$. Also, since $xy \in \ffq$, $x = x(zy+\alpha) = xyz+x\alpha \in \ffq$. 
            \end{proof}
        \item 
            Proof of (b) shows that if $\ffq \lneq R$ such that $\operatorname{rad}(\ffq) = \ffm \in \operatorname{m-Spec}(R)$, then $\ffq$ is primary. \par 
            Alternating proof of (b). Let $\overbar{x},\overbar{y} \in \overbar{R}: = R/\ffq$ such that $\overbar{x} \overbar{y} = 0$. Let $\ffp/\ffq \in \operatorname{Spec}(\overbar{R})$ with $\ffp \in \operatorname{Spec}(R)$ such that $\ffp \supseteq \ffq \supseteq \ffm^{n}$. Then $R \supsetneq \ffp = \operatorname{rad}(\ffq) \supseteq \operatorname{rad}(\ffm^{n}) = \ffm = \in \operatorname{Spec}(R)$. So $\ffp = \ffm$. Hence $\operatorname{Spec}(\overbar{R}) = \{\ffm/\ffq\}$, i.e., $(\overbar{R},\ffm/\ffq)$ is local. If $\overbar{y} \in \ffm/\ffq = \operatorname{Nil}(R/\ffq)$, done. Assume now $\overbar{y} \not \in \operatorname{Nil}(R/\ffq) = \ffm/\ffq$. Then $\overbar{y} \in \overbar{R}^{\times}$. Since $\overbar{x} \overbar{y} = 0$ in $\overbar{R}$, $\overbar{x} = 0$.
        \item Let $p \in \bbZ$ be prime. Then $\langle p \rangle$ is maximal and $\langle p^{n} \rangle$ is primary by (a).
    \end{enumerate}
\end{example}

\begin{example}
    \begin{enumerate}
        \item If $R$ is a UFD and $p \in R$ is prime, then $\langle p^{n} \rangle$ is primary.
        \item Let $R = \frac{k\llbracket X,Y ,Z \rrbracket}{\langle X^{2}-YZ \rangle}$ and $x := \overbar{X} \in R$. Then $x$ is irreducible. Note $R/\langle x \rangle = \frac{k\llbracket X,Y,Z \rrbracket}{\langle X^{2}-YZ \rangle}/\langle x \rangle \cong \frac{k\llbracket Y,Z \rrbracket}{\langle YZ \rangle}$. Let $y := \overbar{Y}, z := \overbar{Z} \in \frac{k\llbracket Y,Z \rrbracket}{\langle YZ \rangle}$. Then $yz = 0$ and $y,z \neq 0$. So $y,z \not \in (0) = \operatorname{rad}(0) = \operatorname{Nil}(R/\langle x \rangle)$. Thus, $\langle x \rangle$ is not primary.
        \item Let $R = k[X_1,\cdots,X_d]$. Then $I = \langle X_{i_1}^{e_1}, \cdots ,X_{i_n}^{e_n} \rangle$ with $e_1,\cdots,e_n \geq 1$ is primary. \par
            Let $J = \langle X_{1}^{e_1},\cdots,X_d^{e_d},f_1,\cdots,f_n \rangle \lneq R$ with $e_1,\cdots,e_d \geq 1$ and $f_1,\cdots,f_n \in R \setminus R^{\times}$. Since $\operatorname{rad}(J) = \langle X_{1},\cdots,X_d \rangle \in \operatorname{m-Spec}(R)$, by Example 4.11(c), we have $J$ is primary.
        \item Let $R = k[X,Y,Z]$. Let $I = \langle X^{2},XY\rangle$. Then $\operatorname{rad}(I) = \langle X \rangle$. Since $XY \in I$ with $X \not\in I$ and $Y \not\in \operatorname{rad}(I)$, we have $I$ is not primary. \par 
            Let $J = \langle X,YZ \rangle$. Then $R/J = \frac{k[X,Y,Z]}{\langle X,YZ \rangle} \cong \frac{k[Y,Z]}{\langle YZ \rangle}$. So similar to (b), we have $J$ is not primary.
    \end{enumerate}
\end{example}

\begin{proof}
    \begin{enumerate}
        \item Let $xy \in \langle p^{n} \rangle$ with $x,y \in R$. If $y \in \operatorname{rad}(\langle p^{n} \rangle) = \langle p \rangle$, then done. Assume $y \not \in \langle p \rangle$. Then $p \nmid y$. Since $xy \in \langle p^{n} \rangle$, $p^{n} \mid xy$. Since $xy$ has a unique factorization, $p$ is irreducible and $p \nmid y$, we have $p^{n} \mid x$, i.e., $x \in \langle p^{n} \rangle$.
        \item [(c)]  Assume by symmetry $I = \langle X_1^{e_1},\cdots,X_n^{e_n} \rangle$. Let $f,g \in R$ such that $fg \in I$. If $f \in I$, then done. Assume $f \not\in I$. Let $f = \sum_{i=1}^{s}a_if_i$ for some $s \geq 1$, $a_i \in R \setminus \{0\}$ and $f_i$ monomial for $i = 1,\cdots,s$ and $g = \sum_{i=1}^{t}b_ig_i$ for some $t \geq 1$, $b_i \in R \setminus \{0\}$ and $g_i$ monomial for $i = 1,\cdots,t$. Since $f \not \in I$, $f_i \not \in I$ for some $i \in \{1,\cdots,s\}$. Let $f = \tilde f + \hat f$, where $\hat f$ are all monomials in $I$ and $\tilde f$ are all monomials not in $I$. Without loss of generality, replace $f$ by $\tilde f$ to assume all monomials of $f$ are not in $I$ since we already have $I \ni fg = \tilde f g + \hat fg$ and $I \ni \hat fg$. Use a monomial ordering, e.g. lexicographical order, asssume $f_s$ is the largest monomial occuring in $f$ and $g_t$ is the largest monomial occuring in $g$. Then $f_sg_t$ is the largest monomial occuring in $fg \in I$. So $f_sg_t \in I$. Since $f_s \not \in I$, $X_i^{e_i} \nmid f_s$ for $i = 1,\cdots,n$. So $g_t$ is not a constant in $R$ and hence $X_j \mid g_t$ for some $j \in \{1,\cdots,n\}$. Then $g_t \in \langle X_1,\cdots,X_n \rangle = \operatorname{rad}(\langle X_1^{e_1},\cdots,X_n^{e_n} \rangle) = \operatorname{rad}(I)$. So $g = \sum_{i=1}^{t-1}b_ig_i + b_tg_t$ with $b_tg_t \in \operatorname{rad}(I)$. Induct on $t$, we have $b_ig_i \in \operatorname{rad}(I)$ for all $i = 1,\cdots,t$. Thus, $g \in \operatorname{rad}(I)$. \qedhere
    \end{enumerate}
\end{proof}

\begin{definition}
    $\ffa \lneq R$ is \emph{reducible} if $\ffa = I \cap J$ for some $I,J \leq R$ with $I \neq \ffa \neq J$. \par
    $\ffa \lneq R$ is \emph{irreducible} if it is not reducible, i.e., if $\ffa = I \cap J$ for some $I,J \leq R$, then $I = \ffa$ or $J = \ffa$.
\end{definition}

\begin{example}
    \begin{enumerate}
        \item If $\ffp \in \operatorname{Spec}(R)$, then $\ffp$ is irreducible.
        \item If $\ffa \leq R$ is primary, then $\ffq$ may not be irreducible.
    \end{enumerate}
\end{example}

\begin{proof}
    \begin{enumerate}
        \item
            Assume $\ffp = I \cap J$ for some $I,J \leq R$. Then $\ffp = I \cap J \supseteq IJ$ by Fact 1.38(f). So $\ffp \in \operatorname{Spec}(R)$, $\ffp \supseteq I$ or $\ffp \supseteq J$. So $I \supseteq I \cap J = \ffp \supseteq I$ or $J \supseteq I \cap J = \ffp \supseteq J$. Hence $\ffp = I$ or $\ffp = J$. Thus, $\ffp$ is irreducible.
        \item 
            Counterexample. In $R = k[X,Y]$, let $\ffa = \langle X^{2},XY,Y^{2} \rangle$, then by Example 4.11(c), $\ffa$ is primary since $\operatorname{rad}(\ffa) = \langle X,Y \rangle \in \operatorname{m-Spec}(R)$, but $\ffa$ is not irreducible since $\ffa = \langle X,Y^{2} \rangle \cap \langle X^{2},Y \rangle$. \qedhere
    \end{enumerate}
\end{proof}

\begin{proposition}
    Let $R$ be noetherian. If $\ffa \lneq R$ is irreducible, then $\ffa$ is primary.
\end{proposition}

\begin{proof}
    Case 1. Assume $\ffa = 0$. Let $x,y \in R$ such that $xy = 0$. If $x = 0$, then done. Assume $x \neq 0$. Note $(0:y) \subseteq (0:y^{2}) \subseteq (0:y^{3}) \subseteq \cdots$. Since $R$ is noetherian, $(0:y^{n}) = (0:y^{n+1})$ for some $n \geq 1$. Let $xs \in \langle x \rangle \cap \langle y^{n} \rangle$ for some $s \in R$. Then $xs = y^{n}t$ for some $t \in R$. So $y^{n+1}t = xys = 0$, i.e., $t \in (0:y^{n+1}) = (0:y^{n})$. Hence $xs = y^{n}t = 0$. So $\langle x \rangle \cap \langle y^{n} \rangle = 0 = \ffa$. Also, since $\ffa$ is irreducible and $\langle x \rangle \neq 0$, we have $\langle y^{n} \rangle = 0$, i.e., $y^{n} = 0$. So $y \in \operatorname{rad}(0) = \operatorname{rad}(\ffa)$. Thus, $\ffa$ is primary. \par 
    Case 2. Assume $\ffa \lneq R$ is arbitrary. To show $\ffa$ is primary, it suffices to show the ideal $0$ is irreducible in $R/\ffa$. Let $I,J \leq R/\ffa$ such that $0 = I \cap J = \frac{\tilde I}{\ffa} \cap \frac{\tilde J}{\ffa} = \frac{\tilde I \cap \tilde J}{\ffa}$ for some $\ffa \leq \tilde I,\tilde J \leq R$ ($\ffa \leq \tilde I \cap \tilde J$). So $\tilde I \cap \tilde J = \ffa$. Also, since $\ffa$ is irreducible, $\tilde I = \ffa$ or $\tilde J = \ffa$. So $I = \frac{\tilde I}{\ffa} = 0$ or $J = \frac{\tilde J}{\ffa} = 0$. Thus, $(0) \leq R/\ffa$ is irreducible.
\end{proof}

\begin{definition}
    Let $\ffa \lneq R$. A \emph{primary decomposition} of $\ffa$ is $\ffa = \bigcap_{i=1}^{n} J_i$ such that $J_i$ is primary for $i = 1,\cdots,n$.
\end{definition}

\begin{theorem}[Noether] 
    Assume $R$ is noetherian and $\ffa \lneq R$. Then $\ffa$ has a primary decomposition.
\end{theorem}

\begin{proof}
    It suffices to show $\ffa = \bigcap_{i=1}^{n} J_i$ for some $n \geq 1$ such that $J_i$ is irreducible for $i = 1,\cdots,n$. Suppose not. Let $\Sigma = \{\ffb \lneq R \mid \ffb \text{ does not have a irreducible decomposition}\}$. Then $\Sigma \neq \emptyset$. Since $R$ is noetherian, $\Sigma$ has a maximal element, say $\ffq$. Then $\ffq = I \cap J$ for some $I \neq \ffq \neq J$. Since $\ffq \subseteq I,J$, we have $\ffq \subsetneq I,J$. Also, since $\ffq$ is maximal, we have $I,J \not \in \Sigma$. So there exists $m \geq n \geq 1$ and $J_1,\cdots,J_m \lneq R$ primary such that $I = \bigcap_{i=1}^{n}J_i$ and $J = \bigcap_{i=n+1}^{m} J_i$. Thus, $\ffq = I \cap J = \bigcap_{i=1}^{m} J_i$, contradicting $\ffq \in \Sigma$.
\end{proof}

\begin{example}
    \begin{enumerate}
        \item Let $R$ be a UFD and $a \in R \setminus \{R^{\times} \cup 0\}$ has a prime factorization $a = up_1^{e_1} \cdots p_n^{e_n}$ with $u \in R^{\times}$, $e_i \geq 1$ and $p_i \nmid p_j$ for $1 \leq i,j \leq n$ with $i \neq j$. Then $\langle a \rangle = \bigcap_{i=1}^{n} \langle p_i^{e_i} \rangle$, a primary decomposition by Example 4.12(a). 
        \item Let $R = k[X_1,\cdots,X_d]$ and $\ffa \lneq R$ be a monomial ideal with a m-irreducible decomposition $\ffa = \bigcap_{i=1}^{n}J_i$ with $J_1,\cdots,J_n$ generated by pure power of variables. Then $J_1,\cdots,J_n$ are primary by Example 4.12(c). So $\ffa = \bigcap_{i=1}^{n} J_i$ is a primary decomposition. Moreover, it is an irreducible decomposition.
        \item Let $R = k[X_1,\cdots,X_d]$ and $\ffa \lneq R$ be a monomial ideal. Let $\ffa = \bigcap_{i=1}^{n} J_i$ be an irredundant m-irreducible decomposition. Then $\ffa$ is primary if and only if $\operatorname{rad}(J_i) = \operatorname{rad}(J_j)$ for $1 \leq i,j \leq n$.
    \end{enumerate}
\end{example}

\begin{proposition}
    If $\ffq \lneq R$ is primary, then $\operatorname{rad}(\ffq) \in \operatorname{Spec}(R)$. In particular, $\operatorname{rad}(\ffq)$ is the unique smallest prime ideal of $R$ containing $\ffq$.
\end{proposition}

\begin{proof}
    Since $\ffq \neq R$, $\operatorname{rad}(\ffq) \neq R$. Let $x,y \in R$ such that $xy \in \operatorname{rad}(\ffq)$. Then $x^{m}y^{m} = (xy)^{m} \in \ffq$ for some $m \geq 1$. Since $\ffq$ is primary, $x^{m} \in \ffq$ or $y^{m} \in \operatorname{rad}(\ffq)$. So $x \in \operatorname{rad}(\ffq)$ or $y \in \operatorname{rad}(\operatorname{rad}(\ffq)) = \operatorname{rad}(\ffq)$ by Fact 1.58(c). Hence $\operatorname{rad}(\ffq) \in \operatorname{Spec}(R)$. The minimality follows from the definition of prime ideal and equivalent definition of primary ideal.
\end{proof}

\begin{definition}
    If $\ffq \lneq R$ is primary and $\ffp = \operatorname{rad}(\ffq)$, then $\ffq$ is $\ffp$-\emph{primary}.
\end{definition}

\begin{example}
    \begin{enumerate}
        \item Let $p \in \bbZ$ be prime. Then $\ffq = \langle  p^{n} \rangle$ is primary with $\operatorname{rad}(\ffq) = \langle p  \rangle \in \operatorname{Spec}(\bbZ)$ for $n \geq 1$.
        \item 
            If $\ffm \in \operatorname{m-Spec}(R)$ and $\ffm^{n} \subseteq \ffq \subseteq \ffm$ for some $m \geq 1$, then $\operatorname{rad}(\ffq) = \ffm \in \operatorname{Spec}(R)$ by Example 4.11(b).
        \item Let $R = k[X_1,\cdots,X_d]$ and $\ffq = \langle X_{i_1}^{e_{1}},\cdots,X_{i_n}^{e_{n}} \rangle$ with $e_i \geq 1$ for $i=1,\cdots,n$, then $\operatorname{rad}(\ffq) = \langle X_{i_1},\cdots,X_{i_n} \rangle \in \operatorname{Spec}(R)$.
    \end{enumerate}
\end{example}

\begin{proposition}
    Let $\ffq_1,\cdots,\ffq_n \lneq R$ be $\ffp$-primary. Then $\bigcap_{i=1}^{n} \ffq_i$ is $\ffp$-primary.
\end{proposition}

\begin{proof}
    Induct on $n$. The base case $n = 2$ is the important case. Note $\operatorname{rad}(\ffq_1 \cap \ffq_2) = \operatorname{rad}(\ffq_1) \cap \operatorname{rad}(\ffq) = \ffp \cap \ffp = \ffp \lneq R$ by Fact 1.58(d). Let $xy \in \ffq_1 \cap \ffq_2$ with $x,y \in R$. If $y \in \ffp = \operatorname{rad}(\ffq_1 \cap \ffq_2)$, then done. Assume $y \not \in \ffp = \operatorname{rad}(\ffq_1)$. Since $xy \in \ffq_1 \cap \ffq_2 \subseteq \ffq_1$ and $\ffq_1$ is primary, we have $x \in \ffq_1$. Similarly, we have $x \in \ffq_2$. So $x \in \ffq_1 \cap \ffq_2$. Hence $\ffq_1 \cap \ffq_2$ is primary.
\end{proof}

\begin{definition}
    A primary decomposition $\ffa = \bigcap_{i=1}^{n}\ffq_i$ is \emph{minimal} if 
    \begin{enumerate}
        \item $\operatorname{rad}(\ffq_i) \neq \operatorname{rad}(\ffq_j)$ for $1 \leq i,j \leq n$ with $i \neq j$,
        \item $\bigcap_{i=1,i \neq j}^{n} \ffq_i \not \subseteq \ffq_j$, i.e., $\ffa \subsetneq \bigcap_{i=1, i \neq j}^{n} \ffq_i$ for $j = 1,\cdots,n$.
    \end{enumerate}
\end{definition}

\begin{example}
    \begin{enumerate}
        \item 
            Let $n \in \bbZ$ and $n = p_1^{e_1} \cdots p_m^{e_m}$ such that $e_1,\cdots,e_m \geq 1$ and $p_1,\cdots,p_m$ are distinct primes. Then the primary decomposition $\langle n \rangle = \bigcap_{i=1}^{m} \langle p_i^{e_i} \rangle$ is minimal.
        \item Let $R = k[X,Y]$. The $\langle X^{2},XY \rangle = \langle X^{2},Y \rangle \cap \langle X \rangle = \langle X^{2},XY,Y^{2} \rangle \cap \langle X \rangle$ are two minimal primary decompositions.
    \end{enumerate}
\end{example}

\noindent \textbf{Notice:} minimal primary decomposition is not necessarily unique up to re-ordering.

\begin{definition}
    Let $\ffa = \bigcap_{i=1}^{n} \ffq_i$ be a minimal primary decomposition such that $\operatorname{rad}(\ffq_i) = \ffp_i$ for $i = 1,\cdots,n$.
    \begin{enumerate}
        \item The \emph{associated primes} of $\ffa$ are $\ffp_1,\cdots,\ffp_n$. Write it as $\operatorname{Ass}_R(\ffa) = \{\ffp_1,\cdots,\ffp_n\}$.
        \item The \emph{minimal (associated) primes} of $\ffa$ are the minimal elements of $\{\ffp_1,\cdots,\ffp_n\}$ w.r.t. $\subseteq$. Write it as $\operatorname{Min}(\ffa) = \operatorname{Min}(\operatorname{Ass}_R(\ffa)) = \operatorname{Min}\{\ffp_1,\cdots,\ffp_n\}$.
        \item The \emph{embedded primes} of $\ffa$ are the non-minimal associated primes of $\ffa$, which are $\operatorname{Ass}_R(\ffa) \setminus \operatorname{Min}(\ffq)$.
    \end{enumerate}
\end{definition}

\begin{example}
    Let $R = k[X,Y]$ and $\ffa = \langle X^{2},XY \rangle$. Then $\operatorname{Ass}_R(\ffa) = \{\langle X \rangle, \langle X,Y \rangle\}$, $\operatorname{Min}(\ffa) = \{\langle X \rangle\}$ and $\langle X,Y \rangle$ is the unique embedded prime of $\ffa$.
\end{example}

\noindent \textbf{Goals:} $\operatorname{Ass}_R(\ffa)$ is independent of the primary decomposition; the $\ffq_i$ such that $\ffq_i \in \operatorname{Min}(\ffa)$ are also independent of the decomposition and $\operatorname{Ass}_R(\ffa) = \operatorname{Ass}_R(R/\ffa)$.

\begin{proposition}
    If $\ffa \leq R$ has a primary decomposition, then $\ffa$ has a minimal primary decomposition.
\end{proposition}

\begin{proof}
    Let $\ffa = \bigcap_{i=1}^{n} \ffq_i$ be a primary decomposition. If $\operatorname{rad}(\ffq_i) = \operatorname{rad}(\ffq_j)$ for some $1 \leq i,j \leq n$ with $i \neq j$, then $\ffq_i \cap \ffq_j$ is $\ffp$-primary by Proposition 4.22, where $\ffp = \operatorname{rad}(\ffq_i)$, so combine $\ffq_i$ and $\ffq_j$ to get a new shorter decomposition, this process terminates in at most $n$ steps. Then without loss of generality, assume $\ffp_i = \operatorname{rad}(\ffq_i) \neq \operatorname{rad}(\ffq_j) = \ffp_j$ for $1 \leq i,j \leq n$ with $i \neq j$. If $\bigcap_{i=1,i\neq j}^{n} \ffp_i \subseteq \ffq_j$ for some $j \in \{1,\cdots,n\}$, then $\ffa = \bigcap_{i=1}^{n} \ffq_i = \bigcap_{i=1,i \neq j}^{n} \ffq_i$, so $\bigcap_{i=1,i \neq j}^{n} \ffq_i$ is a shorter decomposition, the process terminates in at most $n$ steps.
\end{proof}

\begin{proposition}
    Let $\ffa \lneq R$ with minimal primary decomposition $\ffa = \bigcap_{i=1}^{n} \ffq_i$ with $\ffp_i = \operatorname{rad}(\ffp_i)$ for $i = 1,\cdots,n$. Re-order the $\ffq_i$'s such that if necessary to assume without loss of generality, $\operatorname{Min}(\ffa) = \{\ffp_1,\cdots,\ffp_m\}$. Then the irreducible components of $\operatorname{V}(\ffa)$ are $\operatorname{V}(\ffp_1),\cdots,\operatorname{V}(\ffp_m)$.
\end{proposition}

\begin{proof}
    Note $\operatorname{rad}(\ffa) = \operatorname{rad}(\bigcap_{i=1}^{n}\ffq_i) = \bigcap_{i=1}^{n} \operatorname{rad}(\ffq_i) = \bigcap_{i=1}^{n} \ffp_i = \bigcap_{i=1}^{m} \ffp_j$ since for $m < i \leq n$, there exists $j \leq m$ such that $\ffp_i \subseteq \ffp_j$. So by Fact 1.58(g), $\operatorname{V}(\ffa) = \operatorname{V}(\operatorname{rad}(\ffa)) = \operatorname{V}(\bigcap_{j=1}^{m}\ffp_i) = \bigcup_{j=1}^{m} \operatorname{V}(\ffp_j)$. Let $j \in \{1,\cdots,m\}$. Since $\ffp_j \in \operatorname{Spec}(R)$, $\operatorname{V}(\ffp_j)$ is irreducible. For $1 \leq i \leq m$ with $i \neq j$, since $\ffp_i \not\subseteq \ffp_j$, $\operatorname{V}(\ffp_i) \not\supseteq \operatorname{V}(\ffp_j)$. So $\operatorname{V}(\ffp_1),\cdots,\operatorname{V}(\ffp_m)$ are all maximal irreducible subset of $\operatorname{V}(\ffa)$. 
\end{proof}

\begin{proposition}
    Let $\ffq \lneq R$ be $\ffp$-primary and $x \in R$. Then 
    \[
        (\ffq:x) = \left\{
            \begin{array}{ll}
                R &\text{if }x \in \ffq \\
                \ffq &\text{if }x \not \in \ffp \\
                \ffp\text{-primary} &\text{if }x \not \in \ffq \\
            \end{array}
        \right..
    \]
\end{proposition}

\begin{proof}
    If $x \in \ffq$, then $1 \in (\ffq:x)$, so $(\ffq:x) = R$. \par
    Note $(\ffq:x) \supseteq \ffq$ by definition of colon ideal. Let $y \in (\ffq:x)$, then $yx \in \ffq$. Assume $x \not \in \ffp = \operatorname{rad}(\ffq)$. Since $\ffq$ is primary, $y \in \ffq$ or $x \in \operatorname{rad}(\ffq)$. So $y \in \ffq$ and hence $(\ffq:x) \subseteq \ffq$. \par
    Assume $x \not \in \ffq$. If $x \not \in \ffp$, then $(\ffq:x) = \ffq$, which is $\ffp$-primary. So we assume without loss of generality $x \in \ffp \setminus \ffq$. Let $y \in (\ffq:x)$. Then $xy \in \ffq$. Since $\ffq$ is primary, $x \in \ffq$ or $y \in \operatorname{rad}(\ffq) = \ffp$. So $\ffq \subseteq (\ffq:x) \subseteq \ffp$. Then $\ffp = \operatorname{rad}(\ffq) \subseteq \operatorname{rad}(\ffq:x) \subseteq \operatorname{rad}(\ffp) = \ffp$. So $\operatorname{rad}(\ffq:x) = \ffp$. Next, let $ab \in (\ffq:x)$ with $a,b \in R$. If $b \in \operatorname{rad}(\ffq:x)$, then done. Assume $b \not \in \operatorname{rad}(\ffq:x) = \ffp = \operatorname{rad}(\ffq)$. Since $ab \in (\ffq:x)$, $ax \cdot b = abx \in \ffq$. Since $b \not \in \operatorname{rad}(\ffq)$, $ax \in \ffq$. So $a \in (\ffq:x)$. Hence $(\ffq:x)$ is primary.
\end{proof}

\begin{proposition}
    If $\ffa \lneq R$ has a minimal primary decomposition $\ffa = \bigcap_{i=1}^{n}\ffq_i$ such that $\ffp_i = \operatorname{rad}(\ffq_i)$ for $i = 1,\cdots,n$, then $\{\ffp_1,\cdots,\ffp_n\} = \operatorname{Spec}(R) \cap \{\operatorname{rad}(\ffa:x) \mid x \in R\}$. So $\operatorname{Ass}_R(\ffa)$ is independent of the minimal primary decomposition. 
\end{proposition}

\begin{proof}
    Since $(\ffa:x) = (\bigcap_{i=1}^{n} \ffq_i:x) = \bigcap_{i=1}^{n} (\ffq_i:x)$ by Fact 1.54(i), $\operatorname{rad}(\ffa:x) = \operatorname{rad}(\bigcap_{i=1}^{n}(\ffq_i:x)) = \bigcap_{i=1}^{n} \operatorname{rad}(\ffq_i:x) = \bigcap_{i=1, x \not \in \ffq_i}^{n} \ffp_i$ by Proposition 4.29. So $\operatorname{Spec}(R) \cap \{\operatorname{rad}(\ffq:x) \mid x \in R\} \neq \emptyset$. \par 
    ``$\supseteq$''. Let $\ffp \in \operatorname{Spec}(R) \cap \{\operatorname{rad}(\ffa:x) \mid x \in R\}$. Then $\ffp \in \operatorname{Spec}(R)$ and $\ffp = \operatorname{rad}(\ffa:x) = \bigcap_{i=1,x \not \in \ffq_i}^{n} \ffp_i$ for some $x \in R$. By Fact 1.38(f), $\ffp \supseteq \ffp_i$ for some $i \in \{1,\cdots,n\}$, where $x \not \in \ffq_i$. Clearly, $\ffp \subseteq \ffp_i$. So $\ffp = \ffp_i$. \par 
    ``$\subseteq$''. Let $j \in \{1,\cdots,n\}$. Since $\ffa = \bigcap_{i=1}^{n}\ffq_i$ is a minimal primary decomposition, $\bigcap_{i=1,i \neq j}^{n} \ffq_i \not\subseteq \ffq_j$. Then there exists $x \in \bigcap_{i=1,i \neq j}^{n} \ffq_i$ such that $x \not\in \ffq_j$, i.e., $x \in \ffq_i$ for $1 \leq i \leq n$ with $i \neq j$ and $x \not \in \ffq_j$. So $\operatorname{rad}(\ffa:x) = \bigcap_{i=1,x \not \in \ffq_i}^{n} \ffp_i = \ffp_j$. Hence $\ffp_j \in \{\operatorname{rad}(\ffa:x) \mid x \in R\}$.
\end{proof}

\begin{theorem}
    If $R$ is noetherian and $\ffa \lneq R$ with minimal primary decomposition $\ffa = \bigcap_{i=1}^{n} \ffq_i$ with $\ffp_i = \operatorname{rad}(\ffq_i)$ for $i = 1,\cdots,n$, then $\operatorname{Ass}_R(\ffa) := \{\ffp_1,\cdots,\ffp_n\} = \operatorname{Spec}(R) \cap \{(\ffa:x) \mid x \in R\} = \operatorname{Spec}(R) \cap \{\operatorname{Ann}_R(\overbar{x}) \mid \overbar{x} \in R/\ffa\} =: \operatorname{Ass}_R(R/\ffa)$.
\end{theorem}

\begin{proof}
    Proof of the first equality. ``$\supseteq$''. Let $\ffp \in \operatorname{Spec}(R)$ such that $\ffp = (\ffa:x)$ for some $x \in R$. Then $\ffp = \operatorname{rad}(\ffp) = \operatorname{rad}(\ffa:x)$. By Proposition 4.30, $\ffp = \ffp_i$ for some $i \in \{1,\cdots,n\}$. ``$\subseteq$''. Let $j \in \{1,\cdots,n\}$. Since $\ffa = \bigcap_{i=1}^{n}\ffp_i$ is a minimal primary decomposition, $\ffa \subsetneq \bigcap_{i=1, i \neq j}^{n} \ffq_i$. Since $R$ is noetherian, $\ffp_j$ is finitely generated. Also, since $\operatorname{rad}(\ffq_j) = \ffp_j$, there exists $m \geq 1$ such that $\ffp_j^{m} \subseteq \ffq_j$. Let $\ffa_j := \bigcap_{i=1,i \neq j}^{n} \ffq_i$. Then $\ffa_j \ffp_j^{m} \subseteq \ffa_j \cap \ffp_j^{m} \subseteq \ffa_j \cap \ffq_j = \bigcap_{i=1}^{n} \ffq_i = \ffa$. Let $l = \min\{m \geq 1 \mid \ffa_j \ffp_j^{m} \subseteq \ffa\}$. Note $\ffa_j \ffp_j^{0} = \ffa_j \supsetneq \ffa$. Let $x \in \ffa_j \ffp_j^{l-1} \setminus \ffa \subseteq \ffa_j \setminus \ffa$. Then $x \in \ffq_i$ for $1 \leq i \leq n$ with $i \neq j$ and $x \not \in \ffq_j$. So by the proof of Proposition 4.30, $(\ffa:x) \subseteq \operatorname{rad}(\ffa:x) = \ffp_j$. On the other hand, since $x\ffp_j \subseteq \ffa_j\ffp_j^{l-1}\ffp_j = \ffa_j \ffp_j^{l} \subseteq \ffa$, we have $\ffp_j \subseteq (\ffa:x)$. Hence $\ffp_j = (\ffa:x)$.
\end{proof}

\begin{example}
    If $R$ is not noetherian, then $\ffa \lneq R$ may not have a primary decomposition. Let $R = \llC([0,1]) = \{\text{continuous }f: [0,1] \to \bbR\}$ with pointwise operations. Claim. $0 \leq R$ does not have a primary decomposition.
    \begin{enumerate}
        \item For $a \in [0,1]$, define $\Phi_a: R \to R$ by $\Phi_a(f) = f(a)$. Then $\Phi_a$ is a well-defined ring epimorphism. So $\bbR \cong \frac{R}{\ker(\Phi_a)}$. So $\{f \in R \mid f(a) = 0\} = \ker(\Phi_a) \in \operatorname{m-Spec}(R) \subseteq \operatorname{Spec}(R)$.
        \item $0 \not \in \operatorname{Spec}(R)$. For $a \in (0,1)$, $g_a,h_a \in R$ and $g_ah_a = 0$ but $g_a,h_a \neq 0$.
        \item $\operatorname{Nil}(R) = 0$. Proof. Let $f \in \operatorname{Nil}(R)$. So $f^{m} = 0$ for some $m \geq 1$. Then $(f(a))^{m} = 0$ with $f(a) \in \bbR$ for all $a \in [0,1]$. So $f(a) = 0$ for all $a \in [0,1]$. Hence $f= 0$.
        \item 
        \item
            ....Now suppose $0 = \bigcap_{i=1}^{n}\ffq_i$ is a primary decomposition. Assume without loss of generality that the decomposition is minimal. For $i \in \{1,\cdots,n\}$, there exists $f \in R$ such that $\operatorname{Spec}(R) \not \ni (\ffa:f) = \operatorname{rad}(0:f) = \operatorname{rad}(\ffq_i) \in \operatorname{Spec}(R)$, a contradiction. 
        \item Note $0 = \bigcap_{a \in [0,1]} \ker(\Phi_n) = \bigcap_{a \in [0,1]} \smallunderbrace{\{f \in R \mid f(a) = 0\}}_{\in \operatorname{Spec}(R)} = \bigcap_{a \in [0,1] \cap \bbQ} \ker(\Phi_a)$ cannot be pround to a minimal decomposition.
    \end{enumerate}
\end{example}

\begin{proposition}
    If $\ffa \lneq R$ with minimal primary decomposition $\ffa = \bigcap_{i=1}^{n} \ffq_i$ such that $\ffp_i = \operatorname{rad}(\ffq_i)$ for $i = 1,\cdots,n$, then $D = \{x \in R \mid (\ffa:x) \neq \ffa\} = \bigcap_{i=1}^{n} \ffp_i = \bigcup_{\ffp \in \operatorname{Min}(\ffa)} \ffp$.
\end{proposition}

\begin{proof}
    Claim $D = \bigcup_{y \not\in \ffa} \operatorname{rad}(\ffa:y)$. ``$\subseteq$''. ``$\supseteq$''.
\end{proof}

\begin{corollary}
    If $0 \lneq R$ has a mimimal primary decomposition $0 = \bigcap_{i=1}^{n} \ffq_i$ such that $\ffp_i = \operatorname{rad}(\ffq_i)$ for $i = 1,\cdots,n$, then $\operatorname{ZD}(R) = \bigcup_{i=1}^{n} \ffp_i = \bigcup_{\ffp \in \operatorname{Ass}_R(0)}\ffp$.
\end{corollary}

summary:

\begin{example}
    Let $R = \frac{k[X,Y]}{\langle X^{2},XY \rangle} = \frac{k[X,Y]}{\langle X \rangle \cap \langle X^{2},Y \rangle}$.
\end{example}

\begin{proposition}
    Let $\ffa \lneq R_1$ with minimal primary decomposition. $\ffa = \bigcap_{i=1}^{n} \ffq_i$ with $\ffp_i = \operatorname{rad}(\ffq_i)$ for $i = 1,\cdot,n$. Then $\operatorname{Min}(\ffq) = \operatorname{Min}\{\ffp_1,\cdots,\ffp_n\} = \operatorname{Min}(\operatorname{V}(\ffa)) = \operatorname{Min}\{\ffp \supseteq \ffa)\}$. 
\end{proposition}

\begin{proof}
    4.28, $\operatorname{V}(\ffa) = \bigcup_{i=1}^{n} \operatorname{V}(\ffp_i)$. $\ffp_i \in \operatorname{Min}(R)$.
\end{proof}

\begin{lemma}
    Let $U \subseteq R$ be multiplicatively closed and $\ffq \subseteq R$ be $\ffp$-primary. Let $\psi: R \to U^{-1}R$ be the natural ring homomorphism.
    \begin{enumerate}
        \item If $U \cap \ffp \neq \emptyset$, then $U^{-1}\ffq = U^{-1}R$.
        \item If $U \cap \ffp = \emptyset$, then $U^{-1}\ffq \lneq U^{-1}R$ is $U^{-1}\ffp$-primary and $\psi^{-1}(U^{-1}\ffq) = \ffq$.
    \end{enumerate}
\end{lemma}

\begin{proof}
    \begin{enumerate}
        \item 
        \item  \qedhere
    \end{enumerate}
\end{proof}

\begin{theorem}[Second uniqueness theorem]
    Let $\ffa \lneq R$ be minimal primary decomposition $\ffa = \bigcap_{i=1}^{n} \ffq_i$ with $\ffp_i = \operatorname{rad}(\ffq_i)$ for $i = 1,\cdots,n$. 
    \begin{enumerate}
        \item For $\ffp_i \in \operatorname{Min}(\ffq)$: $\ffq_i = \psi^{-1}(\ffa_{\ffp_i})$, where $\psi: R \to R_{\ffp_i}$ and $U = R \setminus \ffp_i$. So $\ffq_i$ is independent of choice of minimal decomposition.
        \item If $\Lambda = \langle \ffp_{i_1},\cdots,\ffp_{i_m} \rangle$ is an ``isolated'' subset of $\operatorname{Ass}(R) = \{\ffp_1,\cdots,\ffp_n\}$, then $\bigcap_{j=1}^{m} \ffq_{i_j} = \psi^{-1}(U^{-1}\ffa)$, where $\psi: R \to U^{-1}R$ and $U = R \setminus \{\ffp_{i_1},\cdots,\ffp_{i_n}\}$. So $\bigcap_{j=1}^{m}\ffq_{i_j}$ is independent of choice of minimal decomposition.
    \end{enumerate}
\end{theorem}


\begin{proof}
    \begin{enumerate}
        \item 
        \item \qedhere
    \end{enumerate}
\end{proof}

\begin{discussion}
    If $\ffm \in \operatorname{m-Spec}(R)$, then $\ffm^{n}$ is $\ffm$-primary for $n \geq 1$ by 4.11. If $\ffp = \langle X_{i_1}, \cdots, X_{i_m} \rangle \lneq k[X_1,\cdots,X_d]$, then $\ffp^{n}$ is $\ffp$-primary for $n \geq 1$.
    \begin{proof}
    \end{proof}
\end{discussion}

\begin{definition}
Let $\ffp \in \operatorname{Spec}(R)$ and $\psi: R \to R_\ffp$. Then for $n \geq 1$, the $n^{th}$ symbolic power of $\ffp$ is $\ffp^{n} = \psi^{-1}((\ffp^{n})_\ffp) = \psi^{-1}\left( \ffp_\ffp \right)^{n})$. 
\end{definition}

\begin{notation}
    $\ffp^{n} \subseteq \ffp^{(n)}$ because by 1.63(a), $\ffp^{n} \subseteq \psi^{-1}((\ffp^{n})_\ffp) = \ffp^{(n)}$.
\end{notation}

\begin{example}
    \begin{enumerate}
        \item 
            Let $\ffm \in \operatorname{m-Spec}(R)$ and $\psi: R \to R_\ffm$. By 4.37(b), $\ffm^{n} = \psi^{-1}((\ffm^{n})_\ffm) = \ffm^{(n)}$. since $\ffm^{n}$ is $\ffm$-primary and $\ffm \cap (R \setminus \ffp) = \emptyset$.
        \item Let $k$ be a field and $\ffp = \langle X_{i_1}, \cdots, X_{i_m} \rangle \lneq k[X_1,\cdots,X_d]$. Then $\ffp^{(n)} = \ffp^{n}$ for $n \geq 1$ (same proof as in (a) since $\ffp^{n}$ is $\ffp$-primary).
        \item Let $R = \frac{k[X,Y,Z]}{\langle XY-Z^{2} \rangle}$ and $\ffp = \langle \overbar{X}, \overbar{Z} \rangle$. $\ffp^{2} = \langle \overbar{X} \rangle \supsetneq \langle \overbar{X}^{2}, \overbar{X}\overbar{Z},\overbar{X} \overbar{Y} \rangle = \langle \overbar{X}^{2}, \overbar{X} \overbar{Z}, \overbar{Z}^{2} \rangle = \ffp^{2}$...
    \end{enumerate}
\end{example}

\begin{proposition}
    If $\ffp \in \operatorname{Spec}(R)$, then $\ffp^{(n)}$ is the ``$\ffp$-primary'' component of $\ffp^{n}$.
\end{proposition}
