\chapter{Modules and Integral Dependence}

\section*{Modules}
\addcontentsline{toc}{section}{Modules}

Let $R$ be a commutative ring with identity.

\begin{definition}
    An \emph{$R$-module} is an additive abelian group $M$ equipped with a scalar multiplication $R \times M \to M$ denoted $(r,m) \to rm$ that is unital, associative and distributive.
    \begin{itemize}
        \item $1m = m$ for all $m \in M$.
        \item $s(rm) = (sr)m$ for all $r,s \in R$ and $m \in M$.
        \item $(r+s)m = rm + sm$ for all $r,s \in R$ and $m \in M$.
        \item $r(m+n) = rm + rn$ for all $r \in R$ and $m,n \in M$.
    \end{itemize}
    \quad (closure) For all $r \in R$ and $m \in M$: $rm \in M$.
\end{definition}

\begin{example}
    \begin{enumerate}
        \item 
            For $n = 1,2,3,\cdots$, let $R^{n} = \left\{\begin{bmatrix}r_1 \\ \vdots \\ r_n\end{bmatrix}\mathrel{\Bigg |} r_1,\cdots,r_n \in R\right\}$ with $s \begin{bmatrix}r_1 \\ \vdots \\ r_n\end{bmatrix} = \begin{bmatrix}sr_1 \\ \vdots \\ sr_n\end{bmatrix}$ for $s \in R$. e.g., $R$ is an $R$-module.
        \item Let $R = \bbZ$: $\bbZ$-module $=$ additive abelian group.
        \item Let $\varphi: R \to S$ be a ring homomorphism. Then $S$ is an $R$-module with $r \cdot s = \varphi(r)s$ for $r \in R$ and $s \in S$.
    \end{enumerate}
\end{example}

\noindent Let $M$ be an $R$-module. 

\begin{definition}
    A \emph{submodule} of $M$ is a subset $N \subseteq M$ such that $N$ is an $R$-module using the operations from $M$.
\end{definition}

\begin{example}
    \begin{enumerate}
        \item If $I \leq R$, then $I$ is a submodule of $R =: M$.
        \item If $R = \bbZ$, submodule $=$ subgroup.
        \item Submodule test. $0 \neq N \subseteq M$ is a submodule of $M$ if and only if $n \pm n' \in N$ for all $n,n' \in N$ and $rn \in N$ for all $r \in R$ and $n \in N$.
        \item If $M_\lambda \subseteq M$ is a submodule for $\lambda \in \Lambda$, then $\bigcap_{\lambda \in \Lambda} M_\lambda \subseteq M$ and $\sum_{\lambda \in \Lambda} M_\lambda \subseteq M$ are submodules.
    \end{enumerate}
\end{example}

\begin{definition}
    If $Y \subseteq M$, 
    \[\langle Y \rangle = R\langle Y \rangle = R(Y) = \bigcap_{Y \subseteq N \subseteq M}N,\]
    intersection of all submodules $N \subseteq M$ such that $Y \subseteq N$. This is the (unique) smallest submodule of $M$ containing $Y$. e.g., for a submodule $N \subseteq M$, $\langle Y \rangle \subseteq N$ if and only if $Y \subseteq N$. \par
    $\langle Y \rangle$ is the \emph{submodule} of $M$ generated by $Y$. $M$ is \emph{finitely generated} if there exists $y_1,\cdots,y_n \in M$ such that $M = \langle y_1,\cdots,y_n \rangle$. 
\end{definition}

\begin{fact}
    \begin{enumerate}
        \item Let $Y \subseteq M$. Then $\langle Y \rangle = \{\sum_{y \in Y}^{\text{finite}} r_yy \mid r_y \in R, \fa y\} = \sum_{y \in Y} \langle y \rangle$.
        \item If $y_1,\cdots,y_n \in M$, then $\langle y_1,\cdots,y_n \rangle = \{\sum_{i=1}^{n} r_iy_i \mid r_1,\cdots,r_n \in R\}$
    \end{enumerate}
\end{fact}

\begin{example}
    Submodules of finitely generated $R$-modules may not be finitely generated. Note $R := k[X_1,X_2,\cdots] = \langle 1 \rangle$ is a finitely generated $R$-module, but $\ffm = \langle X_1,X_2,\cdots \rangle \subseteq R$ is not finitely generated.
\end{example}

\section*{Integral Dependence}
\addcontentsline{toc}{section}{Integral Dependence}

Let $R$ be a nonzero commutative ring with identity. Let $R \subseteq S$ be a subring.

\begin{definition}
    An element $s \in S$ is \emph{integral} over $R$ if there exists a monic $f \in R[X]$ such that $f(s) = 0$, i.e., there exists $n \geq 1$ and $r_0,\cdots,r_{n-1} \in R$ such that $s^{n} + r_{n-1}s^{n-1} + \cdots + r_0 = 0$. \par 
    $S$ is \emph{integral} $R$ if every $s \in S$ is integral over $R$, (or $R \subseteq S$ is an \emph{integral extension}).
\end{definition}

\begin{example}
    \begin{enumerate}
        \item Let $k \subseteq K$ be a field extension. Then $K$ is integral over $k$ if and only if $k \subseteq K$ is an algebraic extension. 
        \item Every $r \in R$ is integral over $R$ since $r$ satisfies $X-r \in R[X]$. 
        \item $\bbZ \subseteq \bbZ[i]$ is an integral extension since $a+bi \in \bbZ[i]$ satisfies $X^{2}-2aX+(a^{2}+b^{2}) \in \bbZ[X]$.
        \item $\bbZ \subseteq \bbQ$. The only $\frac{r}{s} \in \bbQ$ that are integral over $\bbZ$ are the elements of $\bbZ$.
    \end{enumerate}
\end{example}

\begin{proof}
    \begin{enumerate}
        \item[(c)] Let $\frac{r}{s} \in \bbQ$ with $s \neq 0$ and $(r,s) = 1$ be integral over $R$. Then $(\frac{r}{s})^{n} + c_{n-1} (\frac{r}{s})^{n-1} + \cdots + c_1(\frac{r}{s}) + c_0 = 0$ for some $n \geq 1$ and $c_0,\cdots,c_n \in R$. So $\frac{r^{n} + c_{n-1}r^{n-1}s + \cdots + c_1rs^{n-1} + c_0s^{n}}{s^{n}} = 0$, i.e., $r^{n} = -(c_{n-1}r^{n-1}s + \cdots + c_1r s^{n-1} + c_0s^{n}) = -s(c_{n-1}r^{n-1} + \cdots + c_1rs^{n-2} + c_0s^{n-1})$. Hence $s \mid r^{n}$. Since $(r,s) = 1$, $(r^{n},s) = 1$. So $s = \pm 1$. Thus, $\frac{r}{s} = \pm r \in \bbZ$. \qedhere
    \end{enumerate}
\end{proof}

\begin{definition}
    An \emph{intermediate subring} is a subring $T \subseteq S$ such that $R \subseteq T$. (Notice if $R \subseteq T \subseteq S$ is an intermediate subring, then $R \subseteq T$ is a subring.) \par
    Given $y_1,\cdots,y_n \in S$,
    \[R[y_1,\cdots,y_n] = \bigcap_{\substack{R \subseteq T \subseteq S, \\ y_1,\cdots,y_n \in T}}T,\]
    where the intersection is taken over all intermediate subring $T$ such that $y_1,\cdots,y_n \in T$. \par
    $R[y_1,\cdots,y_n]$ is the subring of $S$ generated over $R$ by $y_1,\cdots,y_n$.
\end{definition}

\begin{fact}
    Let $y_1,\cdots,y_n \in S$.
    \begin{enumerate}
        \item $R[y_1,\cdots,y_n] = \{f(y_1,\cdots,y_n) \in S \mid f(y_1,\cdots,y_n) \in R[Y_1,\cdots,Y_n]\}$.
        \item $\psi: R[Y_1,\cdots,Y_n] \to S$ given by $\psi((f(Y_1,\cdots,Y_n))) = f(y_1,\cdots,y_n)$ is a well-defined ring homomorphism, $\im(\psi) = R[y_1,\cdots,y_n]$ and $R[y_1,\cdots,y_n] \cong R[Y_1,\cdots,Y_n]/\ker(\psi) \ni \overbar{Y_i}$. 
        \item For a subring $T \subseteq S$, $R[y_1,\cdots,y_n] \subseteq T$ if and only if $R \subseteq T$ and $y_1,\cdots,y_n \in T$.
    \end{enumerate}
\end{fact}

\begin{example}
    $\bbZ \subseteq \bbC$. Note $\bbZ \subseteq \bbZ[i] \subseteq \bbC$ is an intermediate subring where $\bbZ[i] \cong \bbZ[X]/\langle X^{2}+1 \rangle$.
\end{example}

\begin{proposition}
    Let $s \in S$. Then the followings are equivalent. 
    \begin{enumerate}
        \item[(i)] $s$ is integral over $R$.
        \item[(ii)] $R[s]$ is a finitely generated $R$-module.
        \item[(iii)] There exists an intermediate subring $R \subseteq T \subseteq S$ such that $s \in T$ and $T$ is a finitely generated $R$-module.
    \end{enumerate}
\end{proposition}

\begin{proof}
    ``(i)$\Rightarrow$(ii)''. Assume $s^{n} + r_{n-1} s^{n-1} + \cdots + r_0 = 0$ for some $n \geq 1$ and $r_0,\cdots,r_{n-1} \in R$. Claim. $R[s] = R\langle 1,s,\cdots,s^{n-1} \rangle$. ``$\supseteq$''. It is straightforward. ``$\subseteq$''. It suffices to show $s^{m} \in R\langle 1,s,\cdots,s^{n-1} \rangle$ for $m = n,n+1,\cdots$. Use induction on $m$. Base case: $s^{n} = -\sum_{i=0}^{n-1} r_is^{i} \in R\langle 1,s,\cdots,s^{n-1} \rangle$. Inductive step: asssume $m \geq n+1$ and $s^{k} \in R\langle 1,s,\cdots,s^{n-1} \rangle$ for $0 \leq k \leq m-1$. Then $s^{m} = \sum_{i=0}^{n-1}r_is^{i+m-n} \in R\langle s^{m-n},\cdots,s^{m-1}\rangle \subseteq R\langle 1,s,\cdots,s^{n-1} \rangle$ by inductive hypothesis. \par 
    ``(ii)$\Rightarrow$(iii)''. Use $T = R[s]$. \par 
    ``(iii)$\Rightarrow$(i)''. Assume $T = R\langle y_1,\cdots,y_n \rangle$ for some $y_1,\cdots,y_n \in T$. Since $T$ is a subring containing $R$ and $y_1,\cdots,y_n$, $R[y_1,\cdots,y_n] \subseteq T$ by Fact 5.11(c). So $T = R\langle y_1,\cdots,y_n \rangle \subseteq R[y_1,\cdots,y_n] \subseteq T$, i.e., $R\langle y_1,\cdots,y_n \rangle = T$. For $j = 1,\cdots,n$, since $s \in T$, $sy_j \in T$ and then $\sum_{i=1}^{n}\delta_{ij} sy_i = sy_j = \sum_{i=1}^{n}a_{ij}y_i$, i.e., $\sum_{i=1}^{n} (\delta_{ij}s - a_{ij}) y_i = 0$ for some $a_{1j},\cdots,a_{nj} \in R$. Let $B = (\delta_{ij}s-a_{ij}) \in T^{n \times n}$. Then $B \vec y = 0$. So $(\det(B)\delta_{ij})\vec y = \operatorname{adj}(B)B \vec y = 0$, i.e., $\det(B)y_j = 0$ for $j = 1,\cdots,n$. Since $1 \in T$, there exists $c_1,\cdots,c_n \in R$ such that $1 = \sum_{j=1}^{n}c_jy_j$. So $\det(\delta_{ij}s-a_{ij}) = \det(B) \cdot 1 = \det(B)\sum_{j=1}^{n}c_jy_j = \sum_{j=1}^{n} c_j\det(B)y_j = 0$, i.e., 
    \[0 = \det(\delta_{ij}s-a_{ij}) = 
        \abs{\begin{array}{cccc}
                s-a_{11}& -a_{12} & \cdots & -a_{1n} \\
                -a_{21} & s-a_{22} & \cdots & -a_{2n} \\
                \vdots & \vdots & \ddots \vdots \\
                -a_{n1} & -a_{n2} & \cdots & s-a_{nn}
             \end{array}} = s^{n} + c_{n-1}s^{n-1} + \cdots + c_1s + c_0,\]
             where $c_0,\cdots,c_{n-1} \in R$ since they are built from $a_{ij}\text{'}s \in R$.
\end{proof}

\begin{theorem}
    If $s_1,\cdots,s_n \in S$ are integral over $R$, then $R[s_1,\cdots,s_n]$ is a finitely generated $R$-module. 
\end{theorem}

\begin{proof}
    Claim. given subrings $A \subseteq B \subseteq C$, if $B$ is finitely generated $A$-module and $C$ is finitely generated $B$-module, then $C$ is a finitely generated $A$-module. Assume $B = A\langle b_1,\cdots,b_m \rangle$ and $C = B\langle c_1,\cdots,c_n \rangle$. It suffices to show $C = A \langle b_ic_j \mid i = 1,\cdots,m, j = 1,\cdots,n\rangle$. ``$\supseteq$''. It is straightforward. ``$\subseteq$''. Let $c \in C$. Then $c = \sum_{j=1}^{n} \beta_jc_j$ for some $\beta_1,\cdots,\beta_n \in B$. Note for $j = 1,\cdots,n$, $\beta_j = \sum_{i=1}^{m} \alpha_{ij}b_i$ for some $\alpha_{1j},\cdots,\alpha_{mj} \in A$. So $c = \sum_{j=1}^{n}(\sum_{i=1}^{m}\alpha_{ij}b_i)c_j = \sum_{i=1}^{m} \alpha_{ij}b_ic_j$. Since $s_1$ is integral over $R$, by Proposition 5.13, $R[s_1]$ is a finitely generated $R$-module. Since $s_2$ is integeral over $R$, clearly $s_2$ is integeral over $R[s_1]$ and then $R[s_1,s_2] = R[s_1][s_2]$ is a finitely generated $R[s_1]$-module. So $R[s_1,s_2]$ is a finitely generated $R$-module. Continuing in this fashion, we have $R[s_1,\cdots,s_n]$ is a finitely generated $R$-module.
\end{proof}

\begin{theorem}
    Let $\overbar{R} := \{s \in S \mid s \text{ is integral over }R\}$. Then $R \subseteq \overbar{R} \subseteq S$ is an intermediate subring. So for $s,s' \in S$ integral over $R$, the elements $s \pm s'$ and $ss'$ are integral over $R$.
\end{theorem}

\begin{proof}
    Since $s,s'$ are integral over $R$, $T := R[s,s']$ is a finitely generated $R$-module by Theorem 5.14. Since $s, s' \in T$, $s \pm s',ss' \in T$. So by Proposition 5.13, $s \pm s',ss'$ are integral over $R$. Hence $s \pm s', ss' \in \overbar{R}$. Since $R \subseteq S$ is a subring, $1_S = 1_R \in \overbar{R}$. So by subring test, $\overbar{R} \subseteq S$ is a subring. Furthermore, $R \subseteq \overbar{R}$ similar to Example 5.9(b). 
\end{proof}

\begin{note*}
    Let $s,s' \in R$ be integral over $R$. Assume $s$ satisfies a monic $f \in R[X]$ of degree $m$ and $s'$ satisfies a monic $g \in R[X]$ of degree $n$. Then $s'$ satisfies the monic $g \in R[s][X]$ of degree $n$. So by the proof ``(i)$\Rightarrow$(ii)'' of Proposition 5.13, we have
    \begin{align*}
        R[s,s'] &= R[s][s'] = R[s]\langle 1,s',\cdots,s'^{n-1} \rangle = R\langle 1,s,\cdots,s^{m-1} \rangle \langle 1,s',\cdots,s'^{n-1} \rangle \\
        &= R\langle 1,s',\cdots,s'^{n-1},s,ss',\cdots,ss'^{n-1},\cdots,s^{m-1},s^{m-1}s',s^{m-1}s'^{n-1} \rangle,
    \end{align*}
    which has $mn$ generators. Hence by the proof ``(iii)$\Rightarrow$(i)'' of Proposition 5.13, we have any element in $R[s,s']$, e.g., $s \pm s,ss'$ satisfy monic polynomials of degree $mn$ in $R[X]$.
\end{note*}

\begin{definition}
    $\overbar{R} = \{s \in S \mid s \text{ is integral over }R\}$ is the \emph{integral closure} of $R$ in $S$. \par 
    If $\overbar{R} = S$, then $S$ is integral over $R$. If $\overbar{R} = R$, then $R$ is \emph{integrally closed} in $S$.
\end{definition}

\begin{example}
    \begin{enumerate}
        \item Consider $\bbZ \subseteq \bbQ$, then $\overbar{\bbZ} = \bbZ$, so $\bbZ$ is integrally closed in $\bbQ$.
        \item Consider $\bbZ \subseteq \bbZ[i]$, then $\overbar{\bbZ} = \bbZ[i]$. So $\bbZ[i]$ is integral over $\bbZ$.
        \item Consider $\bbZ \subseteq \bbQ(i)$, then $\overbar{\bbZ} = \bbZ[i]$.
    \end{enumerate}
\end{example}

\begin{definition}
    Let $\varphi: R \to S$ be a ring homomorphism. Then $\varphi$ is \emph{integral} if $\im(\varphi) \subseteq S$ is an integral extension.
\end{definition}

\begin{theorem}
    The followings are equivalent.
    \begin{enumerate}
        \item[(i)]
            $S$ is a finitely generated $R$-module.
        \item[(ii)]
            $S$ is integral over $R$ and there exist $s_1,\cdots,s_n \in S$ such that $S = R[s_1,\cdots,s_n]$.
        \item[(iii)] There exist $s_1,\cdots,s_n \in S$ integral over $R$ such that $S = R[s_1,\cdots,s_n]$.
    \end{enumerate}
\end{theorem}

\begin{proof}
    ``(i)$\Rightarrow$(ii)''. Assume $S = R\langle s_1,\cdots,s_n \rangle$ for some $s_1,\cdots,s_n \in S$. Then by the proof of Proposition 5.13, $S = R[s_1,\cdots,s_n]$. So there exists an intermediate subring $R \subseteq R[s_1,\cdots,s_n]$ such that $s_1,\cdots,s_n \in R[s_1,\cdots,s_n] =: T$ and $T$ is a finitely generated $R$-module. Then by Proposition 5.13, $s_1,\cdots,s_n$ are integral over $R$. Since $\{r \in R \mid r \text{ is integral over }R\} = \overbar{R} \subseteq S$ is a subring by Theorem 5.15, we have $R[s_1,\cdots,s_n] \subseteq \overbar{R}$. So $S = R[s_1,\cdots,s_n]$ is integral over $R$. \par 
    ``(ii)$\Rightarrow$(iii)''. Done. \par 
    ``(iii)$\Rightarrow$(i) by Theorem 5.14.
\end{proof}


\begin{corollary}
    If $R \subseteq S$ and $S \subseteq T$ are integral extensions, then $R \subseteq T$ is an integral extension.
\end{corollary}

\begin{proof}
    Let $t \in T$. Then $t$ is integral over $S$. So $t^n + s_{n-1} t^{n-1} + \cdots + s_0 = 0$ for some $n \geq 1$ and $s_0,\cdots,s_{n-1} \in S$. So $t$ is integral over $R[s_0,\cdots,s_{n-1}]$. Hence $R[s_0,\cdots,s_{n-1},t] = R[s_0,\cdots,s_{n-1}][t]$ is a finitely generated $R[s_0,\cdots,s_{n-1}]$-module by Proposition 5.13. Since $S$ is integral over $R$ and $s_0,\cdots,s_{n-1} \in S$, we have $R[s_0,\cdots,s_{n-1}]$ is a finitely generated $R$-module by Theorem 5.19. Thus, $R[s_0,\cdots,s_{n-1},t]$ is a finitely generated $R$-module by the claim in the proof of Theorem 5.14. Therefore, $t$ is integral over $R$ by Theorem 5.19.
\end{proof}

\begin{corollary}
    If $\overbar{R}$ is an integral closure of $R$ in $S$, then $\overbar{R}$ is integrally closed in $S$, i.e., ${\overbar{\overbar{R}}} = \overbar{R}$.
\end{corollary}

\begin{proof}
    Let $s \in \overbar{\overbar{R}}$. Then $s \in S$ be integral over $\overbar{R}$. So $R \subseteq \overbar{R} \subseteq \overbar{R}[s]$ are integral extensions by Theorem 5.15. Hence $R \subseteq \overbar{R}[s]$ is an integral extension by Corollary 5.20. So $s$ is integral over $R$, i.e., $s \in \overbar{R}$.
\end{proof}

\begin{proposition}
    Let $R \subseteq S$ be an integral extension.
    \begin{enumerate}
        \item If $\ffb \leq S$ and $\ffa = R \cap \ffb$, then $R/\ffa \hookrightarrow S/\ffb$ given by $r+\ffa \mapsto r+\ffq$ is 1-1 and integral. 
        \item If $U \subseteq R$ is multiplicatively closed, then $U^{-1}R \subseteq U^{-1}S$ given by $\frac{r}{u} \mapsto \frac{r}{u}$ is an integral extension.
    \end{enumerate}
\end{proposition}

\begin{proof}
    \begin{enumerate}
        \item 
        \item 
    \end{enumerate}
\end{proof}

\begin{discussion}
    Let $\ffp \in \operatorname{Spec}(R)$. When does there exist $\ffq \in \operatorname{Spec}(S)$ such that $\ffp = R \cap \ffq$? i.e., when is the induced map $\operatorname{Spec}(S) \to \operatorname{Spec}(R)$ surjective? By Cohen-Seidenberg, surjection when $S$ is integral over $R$.
\end{discussion}

\begin{proposition}
    Let $R \subseteq S$ be an integral extension. Then $R$ is a field if and only if $S$ is a field.
\end{proposition}

\begin{proof}
    ``''. \par 
    ``'' nonzero of $S$ translate to the nonzeroness of $R$.
\end{proof}

\begin{example}
    Let $k$ be a field and $S = \frac{k[X]}{(X^{2})}$. Let $x = \overbar{X} \in S$. Then $x$ is integral over $k$ since $x^{2} = 0$. So $S = k[x]$ is integral over $k$. $k$ is a field but $S$ is not a field (conclusion of 5.24 fails if $S$ is not an integral domain).
\end{example}

\noindent Let $0 \neq R \subseteq S$ be an integral extension (no integral extension). (somewhere $\neq 0$).

\begin{corollary}
    Let $\ffq \in \operatorname{Spec}(S)$ and set $\ffp = R \cap \ffq$. Then $\ffp \in \operatorname{m-Spec}(R)$ if and only if $\ffq \in \operatorname{m-Spec}(S)$. 
\end{corollary}

\begin{theorem}
    For $\ffp \in \operatorname{Spec}(R)$, there exists $\ffq \in \operatorname{Spec}(S)$ such that $\ffp = R \cap \ffq$.
\end{theorem}

\begin{proof}
\end{proof}

\begin{proposition}
    Let $\ffq,\ffq' \in \operatorname{Spec}(S)$ such that $R \cap \ffq = R \cap \ffq'$. Then $\ffq \subseteq \ffq'$ if and only if $\ffq = \ffq'$.
\end{proposition}

\begin{proof}
    Set $\ffp = \ffq \cap R = \ffq' \cap R$ and $U = R \setminus \ffq$. 
    \begin{center}
        \begin{tikzcd}
            R \ar[rrr,"\subseteq"] \ar[ddd,"\psi"] & & & S \ar[ddd,"\rho"] \\
            & \ffp \ar[d,mapsto] & \ffq,\ffq' \ar[l,mapsto] \ar[d,mapsto] \\
            & \ffp_\ffp & U^{-1}\ffq, U^{-1}\ffq' \\
            U^{-1}R \ar[rrr,"\subseteq"] & & & U^{-1}S
        \end{tikzcd}
    \end{center}
\end{proof}

\begin{theorem}[Going up]
    Let $\ffp \subseteq \ffp_n$ be a chain in $\operatorname{Spec}(R)$ and $\ffq_1,\cdots,\ffq_m(m < n)$ be a chain in $\operatorname{Spec}(S)$ such that $\ffp_i = R \cap \ffq_i$ for $i = 1,\cdots,m$.
\end{theorem}

\begin{proof}
    By induction on $n-m$. Suffices to consider the case.
\end{proof}

\begin{example}
    \begin{enumerate}
        \item 
        \item $0 \leq 2\bbZ \leq \bbZ \subseteq \bbZ[X]$. 
            \begin{align*}
                \frac{\bbZ[X]}{(2X-1)} &\cong \bbZ_2 = \bbZ[\frac{1}{2}] \subseteq \bbQ \\
                \overbar{X} &\mapsto \frac{1}{2}.
            \end{align*}
            Since $\bbZ[\frac{1}{2}]$ is an integral domain... \par 
            This example also shows need for integral assumption in 5.27 because 
    \end{enumerate}
\end{example}

\begin{proposition}
    If $U \subseteq R$ be multiplicatively closed, then $U^{-1} \overbar{R} = \overbar{U^{-1}R}$. Let $\overbar{R} = \text{integral closure of $R$ in $S$}$. $U^{-1}R \subseteq U^{-1}S$ and $\overbar{U^{-1}R} = \text{integral closure of $U^{-1}R$ in $U^{-1}S$}$, then $\overbar{U^{-1}R} = U^{-1} \overbar{R}$....
\end{proposition}

\begin{proof}
    $U^{-1} \overbar{R} \subseteq \overbar{U^{-1}R}$. Let $s/u \in \overbar{U^{-1}R} \subseteq U^{-1}S$. $0 = (\frac{s}{u})^{n} + (\frac{a_{n-1}}{v_{n-1}})(\frac{s}{u})^{n-1} + \cdots + (\frac{a}{v_1})(\frac{s}{u}) + (\frac{a}{v_0})$ in $U^{-1}S$, where $a_0,\cdots,a_{n-1} \in R$ and $v_0,\cdots,v_{n-1} \in U$. So $v_0 \cdots v_{n-1} \in U$. Multiply by $(uv)^{n}$.
\end{proof}

\begin{definition}
    If $R$ is an integral domain, then $R$ is \emph{integrally closed} (without qualification) if it is integrally closed in $Q(R)$. 
\end{definition}

\begin{example}
    \begin{enumerate}
        \item $\bbZ$ is integrally closed.
        \item Any UFD is integrally closed.
        \item Let $k[X,Y] \supseteq R := k[X^{2},XY,Y^{2}] \cong \frac{k[U,V,W]}{\langle UW-V^{2} \rangle}$...
    \end{enumerate}
\end{example}

\begin{example}
    (Q: $R \subseteq S$ integral and $S$ is noetherian, R is not necessarily noetherian). Let $\overbar{\bbQ} \subseteq \bbC$ algebraic closure, $R := \bbQ + X \overbar{\bbQ}[X] \subseteq \bbQ[X] =: S$. Note $\bbQ$ is not noetherian since $[\overbar{\bbQ}:\bbQ] = \infty$ and $R \subseteq S$ is an integral extension since $\overbar{\bbQ}$ is algebraically over $\bbQ$.
\end{example}

\begin{lemma}
    If $R$ is an integral domain, then $R = \bigcap_{\ffm \in \operatorname{m-Spec}(R)} R_\ffm \subseteq Q(R)$. 
\end{lemma}

\begin{proof}
    Since $R \subseteq R_\ffm$ for $\ffm \in \operatorname{m-Spec}(R)$, $R \subseteq \bigcap_{\ffm \in \operatorname{m-Spec}(R)}R_\ffm$. Let $x \in \bigcap_{\ffm \in \operatorname{m-Spec}(R)}R_\ffm$. Let $I = \{r \in R \mid rx \in R\} = (R:_Rx) \leq R$. By localization chapter, $I_\ffm = (R:_Rx)_\ffm = (R_\ffm:_{R_\ffm}x) = R_\ffm$ for $\ffm \in \operatorname{m-Spec}(R)$. So $I \neq \ffm$ for $\ffm \in \operatorname{m-Spec}(R)$. Hence $I = R$, i.e., $1 \in I = (R:_Rx)$. Thus, $x = 1 \cdot x \in R$. ($R_\ffm$ maximal localization.)
\end{proof}

\begin{recall*}
    An integral domain $R$ is \emph{integrally closed} if it is integral closed in $Q(R)$.
\end{recall*}

\begin{proposition}[being integrally closed is a ``local condition'']
    Let $R$ be an integral domain. Then the followings are equivalent.
    \begin{enumerate}
        \item[(i)] $R$ is integrally closed.
        \item[(ii)] 
        \item[(iii)]
    \end{enumerate}
\end{proposition}

\begin{definition}
    Let $\ffa \leq R \subseteq S$. $s \in S$ is \emph{integral over $\ffa$} if $s$ satisfies $f(X) = X^{n} + a_{n-1}X^{n-1} + \cdots + a_0$ for some $n \geq 1$ and $a_0,\cdots,a_{n-1} \in \ffa$. \par 
    The \emph{integral closure} of $\ffa$ in $S$ is 
    \[\overbar{\ffa} = \{s \in S \mid s \text{ is integral over $\ffa$}\}.\]
\end{definition}

\begin{warning}
    There exists another notion of integral closure of an ideal.
\end{warning}

\begin{lemma}
    Let $\ffa \leq R \subseteq S$ and let $\overbar{R}$ be the integral closure of $R$ in $S$: $\ffa \leq R \subseteq \overbar{R} \subseteq S$. Then $\overbar{\ffa} = \operatorname{rad}(\ffa \overbar{R})$. So $\overbar{\ffa}$ is closed under sums and products.
\end{lemma}

\begin{proposition}
    Let $R \subseteq S$ be a subring such that $S$ is an integral domain. Assume $R$ is integrally closed in $Q(R)$ and let $s \in R$ be integral over $\ffa \leq R$, say $s$ satisfies $f(X) = X^{n} + a_{n-1}X^{n-1} + \cdots + a_0 \in R[X]$ for some $a_0,\cdots,a_{n-1} \in \ffa$. Then $s$ is algebraic over $Q(R) =: K$. ......
\end{proposition}

\begin{theorem}[Going down theorem]
    Let $R \subseteq S$ be an integral extension such that $S$ is an integral domain and $R$ is an integrally closed in $\operatorname{Q}(R)$. Let $\ffp_1 \supseteq \ffp_2 \supseteq \cdots \supseteq \ffp_n$ be a chain in $\operatorname{Spec}(R)$ and $\ffq_1 \supseteq \ffq_2 \supseteq \cdots \supseteq \ffq_m$ ($m < n$) be a chain in $\operatorname{Spec}(S)$ such that $\ffq_i \cap R = \ffp_i$ for $i = 1,\cdots,m$. Then there exists a chain $\ffq_m \supseteq \ffq_{m+1} \supseteq \cdots \supseteq \ffq_n$ in $\operatorname{Spec}(S)$ such that $\ffq_i \cap R = \ffp_i$ for $i = 1,\cdots,n$. 
\end{theorem}

\begin{proof}
    As in the going up, assume without loss of generality that $m = 1$ and $n = 2$. 
    \begin{center}
        \begin{tikzcd}
            R \ar[rrr,"\subseteq"] \ar[dd] & & & S \ar[dd] \\
            & R \ar[r,mapsto] & SSS \\
            R_\ffp \ar[rrr,dashed,"\ex !"] & & & S_\ffq 
        \end{tikzcd}
    \end{center}
\end{proof}

